\documentclass{article}
\usepackage[T1]{fontenc}

\newcommand*{\titleGM}{\begingroup % Create the command for including the title page in the document
\hbox{ % Horizontal box
\hspace*{0.05\textwidth} % Whitespace to the left of the title page
\rule{1pt}{\textheight} % Vertical line
\hspace*{0.05\textwidth} % Whitespace between the vertical line and title page text
\parbox[b]{0.75\textwidth}{ % Paragraph box which restricts text to less than the width of the page

{\noindent\Huge\bfseries So You Want to\\[0.5\baselineskip] Buy a Defender\ldots}\\[2\baselineskip] % Title
{\large \textit{A buyer's primer}}\\[4\baselineskip] % Tagline or further description
{\Large \textsc{chris snell}}\\ % Author name
{\small \textit{DiscoWeb.org}}
\vspace{0.5\textheight} % Whitespace between the title block and the publisher
}}
\endgroup}


\begin{document} 
\titleGM
\newpage
\tableofcontents
\newpage
\section{Do you \textit{really} want a Defender?}
\subsection{The Good, The Bad, and The Total Rip-off}
 The Land Rover Defender may be the most capable and functional mass-production vehicle ever manufactured.  They are phenomenally agile off-road, tough as nails, and can be maintained by an owner of modest mechanical aptitude using a basic set of tools.  They're also beautiful, with a simple design that looks proper in nearly any setting, from a muddy river crossing in Sub-Saharan Africa to a Nantucket driveway and everything in between.  They've transported soldiers, farmers, explorers, and royalty; driving one says a lot about the person behind the wheel.  If you're ready for an unconventional, sometimes uncomfortable, but always timeless ride, the Defender might just be the truck for you.
\\*
\\*
 Buying a Defender might also be the worst decision you've ever made.  They can be money pits and maintenance nightmares, spending more time in the shop than in your driveway.  They can be noisy and uncomfortable to drive, oven-like in the summer and bone-chilling cold in the winter.  Rust is a perpetual problem and it attacks some of the most difficult to repair/replace components of the vehicle--the frame and the bulkhead--with ferocity.  
\\*
\\*
 It can be exceptionally daunting for someone who is new to Defenders to make an informed purchase decision.  The newest Defender available on the US market is 18 years old; most are well over 25 years old and showing their age.  United States Customs and EPA regulations permit the importation of conforming\footnote{See ``What''s Legal'' thread on DefenderSource.com, http://goo.gl/7YULH7} 25+ year-old trucks and with many older Defenders now meeting the requirements, there has been a surge in popularity of these trucks in the U.S. and a corresponding surge in their importation.  As you might expect, there has also been a surge in shady hucksters who are looking to rip off the unsuspecting and inexperienced Defender purchaser.  The variety tactics and techniques used by these thieves is staggering.  Here are just a few of the scams we've seen perpetrated against new Defender purchasers: 

\begin{itemize}
 \item Photos in for-sale ads that have been angled to hide major damage and decay
 \item Frames so rusty that they crumble when poked with a finger, but camouflaged by fresh paint applied over the rust
 \item Dealers who claim to be the ``premier importer of Defenders'' but who have a track record of ripping off unsuspecting new Defender purchasers
 \item Cratered engines advertised as ``runs great''
 \item Non-importable configurations advertised as ``ready to import'', only to get seized by U.S. Customs upon arrival
 \item Rat's nest electrical nightmares
\end{itemize}

This document was put together by members of the community in an attempt to educate you and hopefully help you avoid the common traps that befall so many of our new members.

\subsection{On Reliability}
Contrary to popular belief, the Defender is not an \textit{unreliable} truck.  These trucks have proven themselves time and time again in grueling around-the-world expeditions.  Properly maintained, they can be every bit as reliable as a Toyota or Mercedes.  To achieve this reliability, you will need to purchase a truck in good condition (or be prepared for a costly and time-consuming total overhaul) and you will need to maintain it.  We will cover purchasing in more detail in later chapters but for now, just understand that you cannot expect to buy a 27 year-old Defender for \$6,000 USD and expect it to be as reliable as your 2010 Toyota Tacoma.  Good, reliable Defenders are not cheap.  You're going to either spend the money initially on a good truck, or you're going to spend it on repairs soon after your purchase.  If you aren't careful and make a poor purchasing decision, you may spend a lot on your truck purchase \textit{and} spend it in repairs soon after.

\subsection{Maintenance}
Most Defenders imported today are 25+ years old.  As with any 25 year-old vehicle, there are a number of age-related issues that must be attended to either pre-purchase or post-purchase.  These include:

\begin{itemize}
 \item Electrical wiring that has outgassed and become brittle, cracking and potentially causing electrical faults
 \item Worn-out mechanicals like clutches, gearbox syncros, axle splines, etc.
 \item Dry-rotted rubber gaskets and seals that lead to water ingress, fluid leaks, or even mechanical failure.
\end{itemize}

Some of these items--like an oil pan gasket, for example--can be addressed by the owner in his/her own garage with basic tools.   Others--like a clutch--will require extensive repair by a skilled mechanic.  Can you afford to address these inevitable issues, even if they cost \$2-3,000 USD?

Once you've addressed the age-necessitated maintenace on your truck, you will need to attend to the regular time- and mileage-dependent maintenance.  This includes things like fluid changes, tires, driveshaft lubrication, etc.  This maintenance can be easily performed by the average owner with some instruction and some simple tools.  If you don't want to work on your truck, or perhaps can't because of where you live, you will need to pay someone to do these things for you.

\subsection{Defenders and Families}
If you are buying a Defender with the intention of using it to haul your spouse and children on vacation, you should think carefully about that decision.  For starters, the Defender is built with frame-on-body construction; the cab and tub (the rear part of the body) are built from thin aluminum panels on  steel ribs.   In most spots, there is less than a millimeter of aluminum between the passengers and the world outside.  These 25 year-old trucks have not been crash-tested by the NHTSA and there's no ``star rating'' that describes their crash-worthiness.  Some Defenders have roll cages that offer a modicum of roll-over protection to the passengers, but many do not.   There are no airbags in these trucks; most of them don't even have anti-lock braking systems.  In short, these are not safe vehicles for the highway.  When we drive our trucks, we accept a certain amount of risk in doing so.  Is this something that you want to put your family in?

\section{The Defender Taxonomy}
\subsection{Model years}
\subsection{Body styles}
\subsection{Chassis lengths}
\subsection{Factory-supplied motor and gearbox configurations, by year}

\section{How to Find a Good Truck}
\subsection{How to do a mechanical and structural evaluation}
\subsection{How to find a good importer or reseller}
\subsubsection{Tell-tale signs of a good importer/reseller}
\subsubsection{Tell-tale signs of a bad importer/reseller}

\section{Buying a Truck in the U.S. vs. Importing}
\subsection{Overview of importing regulations - what's importable, what's not}
\subsection{Links to relevant regulatory agency websites}
\subsection{State-specific gotchas}

\section{References}

\end{document}
